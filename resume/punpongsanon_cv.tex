% E. Dunham -- Resume
% Contents Copyright (C) 2014 - 2016, E. Dunham

% LaTeX code for rendering the resume is distributed under the MIT license.
% See LICENSE.txt. It means you can use the code for whatever you want,
% including your own resume, but I'm not liable if it catches your computer on
% fire.

% Template originally developed by E. Dunham
% https://github.com/edunham/resume/blob/master/resume.tex

\documentclass[11pt]{article} % Set default font size to 11pt
% E. Dunham -- Resume
% Contents Copyright (C) 2014 - 2016, E. Dunham

% LaTeX code for rendering the resume is distributed under the MIT license.
% See LICENSE.txt. It means you can use the code for whatever you want,
% including your own resume, but I'm not liable if it catches your computer on
% fire.

% Template originally developed by E. Dunham
% https://github.com/edunham/resume/blob/master/resume.tex

\usepackage[normalem]{ulem} % for the underlines
\usepackage[compact]{titlesec} % Shrink default spacings
\usepackage{tabto} % For aligning skills section

%
\usepackage{xparse}
\ExplSyntaxOn%
\seq_new:N \l_local_enum_seq

\newcommand{\listofpublication}[1]
{%
	\seq_set_from_clist:Nn \l_local_enum_seq {#1}%
}
\newcommand{\itemofpublication}
{%
	\int_zero:N \l_tmpa_int
	\seq_map_inline:Nn \l_local_enum_seq {%
		\int_incr:N \l_tmpa_int% Increase the counter
		\item ##1
		% Check whether the list has reached the end -- if so, use '.' instead of ','
		%\int_compare:nNnTF 
		%{ \l_tmpa_int } < {\seq_count:N \l_local_enum_seq} 
		%{,} {.}
	}
}
\ExplSyntaxOff

\NewDocumentCommand{\publicationthelist}{+m}
{%
	\listofpublication{#1}%
	\begin{enumerate}
		\itemofpublication%
	\end{enumerate}
}
%

\textwidth=7in
\textheight=10.5in
\topmargin -1in % Reclaim the default whitespace from top of page
\oddsidemargin -.25in % Reclaim whitespace on left, make it look centered
\pagenumbering{gobble} % Don't number pages
\setlength{\parindent}{0pt} % Don't indent paragraphs
\newcommand{\heading}[1]{
%    \section*{\centering\uline{\hfill #1 \hfill }} % Center the headings
    \section*{\uline{\hfill #1 }} % Right-align the headings
}
\newcommand{\squish}{
    \setlength{\itemsep}{0.5pt}
    \setlength{\parskip}{0pt} % tweak parskip value to adjust total height
    \setlength{\parsep}{0.5pt}
}
\newcommand{\when}[1]{ % naming this 'date' would conflict with builtins
    \hfill \texttt{#1}
}
\newcommand{\experience}[3]{ % place, optional title, date
    \ifx&#2&
        \item[{#1}]
    \else
        \item[{#1}, \emph{#2}]
    \fi
    \when{#3}
}
\newcommand{\contact}[3]
{
    { \large \texttt{#1} | \texttt{#2} \\ \emph{#3} }
}
\newcommand{\skill}[2]{
    \textbf{#1} \tabto{2.5in} #2
}
% Write C++ all fancy-like
% http://www.parashift.com/c++-faq-lite/latex-macros.html
\newcommand{\CPP}
{
    C\hspace{-.05em}\raisebox{.4ex}{\tiny\bf +}\hspace{-.10em}\raisebox{.4ex}{\tiny\bf +}
}
\newcommand{\publication}[4]
{
	{ {#1}. {`#2'}, \emph{#3}, {#4} }
}

\begin{document}

{{\Huge \bf Parinya Punpongsanon}}
\smallskip

\contact{parinya@mit.edu}
        {http://punpongsanon.info}
        {MIT CSAIL, 32 Vassar St. 32-208, Cambridge, MA 02139}

%%%%%%%%%%%%%%%%%%%%%%%%%%%%%%%%%%%%%%%%%%%%%%%%%%%%%%%%%%
\heading{Employment}
%%%%%%%%%%%%%%%%%%%%%%%%%%%%%%%%%%%%%%%%%%%%%%%%%%%%%%%%%%

\begin{description}
	\squish
	\experience{Postdoctoral Associate}
	{Massachusetts Institute of Technology}
	{01/2017 - Present}
	
	Human-Computer Interaction Engineering (HCIE) Group\\
	Computer Science and Artificial Intelligence Laboratory (CSAIL)\\
	Under advisory of \textit{Prof. Stefanie Mueller}
	
	\experience{Postdoctoral Fellow}
	{Osaka University}
	{10/2016 - Present}
	
	Intelligence Sensing Group\\
	Graduate School of Engineering Science\\
	Sponsors by Japan Society for the Promotion of Science (JSPS)\\
	Under advisory of \textit{Prof. Kosuke Sato}
	
\end{description}

%%%%%%%%%%%%%%%%%%%%%%%%%%%%%%%%%%%%%%%%%%%%%%%%%%%%%%%%%%%%%%
\heading{Education}
%%%%%%%%%%%%%%%%%%%%%%%%%%%%%%%%%%%%%%%%%%%%%%%%%%%%%%%%%%%%%%

\begin{description}
	\squish
	\experience{Ph.D. in Engineering}
	{Osaka University, Japan}
	{September 2016}
	
	System Innovation, Graduate School of Engineering Science,\\
	Under advisory of \textit{Prof. Kosuke Sato} and \textit{Prof. Daisuke Iwai}
	
	\experience{Bachelor of Science}
	{King Mongkut's University of Technology, Thailand}
	{April 2010}
	
	School of Computer Science and Information Technology,\\
	MAJOR GPA: 3.63/4.00 (First Class Honor)
	
\end{description}

%%%%%%%%%%%%%%%%%%%%%%%%%%%%%%%%%%%%%%%%%%%%%%%%%%%%%%%%%%
\heading{Experience}
%%%%%%%%%%%%%%%%%%%%%%%%%%%%%%%%%%%%%%%%%%%%%%%%%%%%%%%%%%

\begin{description}
	\squish
	\experience{Visiting Researcher}
	{Telecom ParisTech (Universite Paris-Saclay)}
	{12/2013 - 02/2014}
	
	Computer Graphics Group\\
	Collaborated under project 'Lazy 3D Navigation using Non-Critical Body Interaction'\\
	Under advisory of \textit{Prof. Tamy Boubekeur}
	
	\experience{Exchange Student}
	{Fukui University}
	{10/2008 - 09/2009}
	
	Human and Computational Intelligence System Laboratory\\
	School of Engineer\\
	Sponsors by Japan Society for the Promotion of Science (JSPS)\\
	Under advisory of \textit{Prof. Yasuhiro Ogoshi}
	
\end{description}

%%%%%%%%%%%%%%%%%%%%%%%%%%%%%%%%%%%%%%%%%%
\heading{Grants and Awards}
%%%%%%%%%%%%%%%%%%%%%%%%%%%%%%%%%%%%%%%%%%

\begin{description}
	\squish
	\experience{Best Student Paper}
	{IEEE Kansai Section}
	{2017}
	
	\experience{Best Student Volunteer}
	{ACM UIST 2016}
	{2016}
	
	\experience{Grant}
	{JSPS Research Fellow}
	{2016}
	
	\experience{Best Paper}
	{IEEE 3DUI 2015}
	{2015}
	
	\experience{Best Student Volunteer}
	{ACM SIGGRAPH Asia 2014}
	{2014}
	
	\experience{Best Presentation}
	{Korea-Japan Workshop on Mixed Reality 2013}
	{2013}
	
	\experience{Grant}
	{MEXT Scholarship (Oct. 2011 - Sep. 2016)}
	{2011}
	
	\experience{$1^{st}$ Class Honor}
	{King Mongkut's University of Technology Thonburi}
	{2010}
	
	\bigskip
	
\end{description}

%%%%%%%%%%%%%%%%%%%%%%%%%%%%%%%%%%%%%%%%%%
\heading{Skills}
%%%%%%%%%%%%%%%%%%%%%%%%%%%%%%%%%%%%%%%%%%

\skill{Software}
{Python, C/\CPP, HTML/CSS/Javascript, MATLAB, OpenCV}

\skill{Hardware}
{Projector-Camera system, Laser Cutter, 3D Printer}

\pagebreak

%%%%%%%%%%%%%%%%%%%%%%%%%%%%%%%%%%%%%%%%%%
\heading{Selected Publication}
%%%%%%%%%%%%%%%%%%%%%%%%%%%%%%%%%%%%%%%%%%

\textbf{JOURNALS}
\publicationthelist
{
	\publication{\underline{\texttt{Parinya Punpongsanon}}, Emilie Guy, Daisuke Iwai, Kosuke Sato, and Tamy Boubekeur}
				{Extended LazyNav: Virtual 3D Ground Navigation for Large Displays and Head-Mounted Displays}
				{IEEE Transactions on Visualization and Computer Graphics}
				{Vol. 23, No. 8, pp. 1952-1963.}
				{August 2017.},
	\publication{\underline{\texttt{Parinya Punpongsanon}}, Daisuke Iwai, and Kosuke Sato}
				{SoftAR: Visually Manipulating Haptic Softness Perception in Spatial Augmented Reality}
				{IEEE Transactions on Visualization and Computer Graphics}
				{Vol. 21, No. 11, pp. 1279-1288.}
				{November 2016.},
	\publication{\underline{\texttt{Parinya Punpongsanon}}, Daisuke Iwai, and Kosuke Sato}
				{Projection-based Visualization of Tangential Deformation of Nonrigid Surface by Deformation Estimation Using Infrared Texture}
				{Springer: Virtual Reality}
				{Vol. 19, No. 1, pp. 45-56.}
				{March 2015.}
}

\textbf{CONFERENCE PAPERS AND NOTES}
\publicationthelist
{
	\publication{Emilie Guy, \underline{\texttt{Parinya Punpongsanon}}, Daisuke Iwai, Kosuke Sato, and Tamy Boubekeur}
	{LazyNav: 3D Ground Navigation with Non-Critical Body Parts}
	{In Proceedings of IEEE Symposium on 3D User Interfaces (3DUI)}
	{pp. 43-50,}
	{2015.},
	\publication{\underline{\texttt{Parinya Punpongsanon}}, Emilie Guy, Tamy Boubekeur, Daisuke Iwai, and Kosuke Sato}
	{Ground Navigation in 3D Scene using Simple Body Motions}
	{In Proceedings of International Conference on Artificial Reality and Telexistence and Eurographics Symposium on Virtual Environments (ICAT-EGVE)}
	{pp. 19-20,}
	{2014.},
	\publication{\underline{\texttt{Parinya Punpongsanon}}, Daisuke Iwai, and Kosuke Sato}
	{SoftAR: Visually Manipulating Haptic Softness Perception in Spatial Augmented Reality}
	{In Proceedings of IEEE Symposium on Mixed and Augmented Reality (ISMAR)}
	{pp. 1279-1288,}
	{2016.},
	\publication{\underline{\texttt{Parinya Punpongsanon}}, Daisuke Iwai, and Kosuke Sato}
	{A Preliminary Study on Altering Surface Softness Perception using Augmented Color and Deformation}
	{In Proceedings of IEEE Symposium on Mixed and Augmented Reality (ISMAR)}
	{pp. 301-032,}
	{2014.},
	\publication{\underline{\texttt{Parinya Punpongsanon}}, Daisuke Iwai, and Kosuke Sato}
	{DeforMe: Projection-based Visualization of Deformable Surfaces using Invisible Textures}
	{In Proceedings of ACM SIGGRAPH Asia (Emerging Technologies)}
	{Article 8,}
	{2013.},
	\publication{\underline{\texttt{Parinya Punpongsanon}}, Daisuke Iwai, and Kosuke Sato}
	{Infrared-based Tangential Deformation Estimation Technique}
	{In Proceedings of the $6^{th}$ Thailand-Japan International Academic Conference (TJIA)}
	{Article 3,}
	{2013.},
}

%%%%%%%%%%%%%%%%%%%%%%%%%%%%%%%%%%%%%%%%%%
\heading{Invited Talks}
%%%%%%%%%%%%%%%%%%%%%%%%%%%%%%%%%%%%%%%%%%

\begin{description}
	\squish
	\experience{University of Tokyo, Japan}
           {}
           {2016}

    `Projection-based Mixed Reality for Deformable Objects'

\experience{The $19^{th}$ Meeting on Image Recognition and Understanding (MIRU 2016), Japan}
           {}
           {2016}
           
	`SoftAR: Visually Manipulating Haptic Softness Perception in Spatial Augmented Reality'

\experience{IEEE TVCG VR/AR Special Session}
           {ACM SIGGRAPH 2016, USA}
           {2016}

    `SoftAR: Visually Manipulating Haptic Softness Perception in Spatial Augmented Reality'

\experience{The $18^{th}$ Annual Meeting on Virtual Reality in Japan}
		   {3DUI Top Conference, Japan}
		   {2015}

	`LazyNav: 3D Ground Navigation with Non-Critical Body Parts'

\experience{The $6^{th}$ Korea-Japan Workshop on Mixed Reality, Japan}
		   {}
		   {2013}

	`Projection-based Mixed Reality for Deformable Surfaces'

\end{description}

\pagebreak

%%%%%%%%%%%%%%%%%%%%%%%%%%%%%%%%%%%%%%%%%%%%%%%%%%%%%%%%%%
\heading{Academic Service}
%%%%%%%%%%%%%%%%%%%%%%%%%%%%%%%%%%%%%%%%%%%%%%%%%%%%%%%%%%

\skill{Organization Committee}
{}

UIST 2017 (Documentation Chair) $\bullet$ SCF 2017 (Local Arrangement Chair) $\bullet$ CHI 2017 (Session Chair) $\bullet$ SUI 2016 (Documentation Chair) $\bullet$ VRSJ 2016 (Design Chair) $\bullet$ ICAT-EGVE 2015 (Design Chair)

\skill{Peer-Reviewer}
{}

IEEE ISMAR (2017, 2016) $\bullet$ ACM SIGGRAPH/SIGGRAPH Asia (2017, 2016) $\bullet$ ACM SUI (2016) $\bullet$ ACM UIST (2017, 2016, 2015) $\bullet$ ACM VRST (2017, 2016, 2015) $\bullet$ ACM HRI (2017, 2016, 2015)

\skill{Student Volunteer}
{}

ACM UIST 2016 $\bullet$ ACM UbiComp 2015 $\bullet$ ACM SIGGRAPH Asia 2014 $\bullet$ ACM Multimedia 2012

%%%%%%%%%%%%%%%%%%%%%%%%%%%%%%%%%%%%%%%%%%%%%%%%%%%%%%%%%%
	\bigskip
	\hfill \small September 2017
%%%%%%%%%%%%%%%%%%%%%%%%%%%%%%%%%%%%%%%%%%%%%%%%%%%%%%%%%%

\end{document}
